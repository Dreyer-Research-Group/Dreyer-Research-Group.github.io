\documentclass[11pt]{article}

%\usepackage[cm]{fullpage}
\usepackage[lmargin=0.75in,rmargin=0.75in,
            tmargin=0.5in,bmargin=0.75in]{geometry}


\usepackage[parfill]{parskip}

\usepackage{hyperref}

%\usepackage{palatino}

% URLs (special font for monospace)
\usepackage{inconsolata}
\usepackage[T1]{fontenc}

%\usepackage[defaultsans]{cantarell}
%\usepackage[T1]{fontenc}

\usepackage[small,compact]{titlesec}
\titlespacing{\subsection}{0pt}{*1}{-0.5\parskip}
\titleformat*{\subsection}{\sffamily\bfseries}

%\usepackage{fancyhdr}

%\pagestyle{fancy}

%\fancyhead{}
%\fancyfoot[LO,LE]{\sffamily \footnotesize PHY 604/Spring 2017---revision 1.05 (2017-09-19)}
%\fancyfoot[CO,CE]{\thepage}
%\renewcommand{\headrulewidth}{0.0pt}
%\renewcommand{\footrulewidth}{0.0pt}

\newenvironment{itemsquish}
  { \begin{itemize}
    % set spacing between items
    \addtolength{\itemsep}{-0.25\baselineskip}
    % set spacing between lines
    \addtolength{\baselineskip}{-0.25\baselineskip} }
  { \end{itemize} }



\begin{document}

\begin{center}
{\LARGE \sffamily \bfseries PHY 555: Solid State Physics I} \\[3mm]
{\em  Instructor:} Cyrus Dreyer, Physics B141, cyrus.dreyer@stonybrook.edu \\
{\em Web:}\/ \url{https://dreyer-research-group.github.io/phy555_fall2022.html}
\end{center}

\subsection*{Office Hours:}

\textit{(Tentatively:)} Mondays, 10:08am-11:00am; Wednesdays, 10:08am-12:00pm

\subsection*{Texts:}

The main textbook that we will use is \textit{Solid State Physics}, by Giuseppe Grosso and Giuseppe Pastori Parravicini, any edition is fine (Elsevier,  isbn: 9780080481029). The main reason this text was chosen (instead of the more classic ones below), is that it provides more modern topics and treatment of solid-state physics. 

Other useful references (optional):
\begin{itemsquish}
\item {\em Solid State Physics}, N.W.~Ashcroft and  N.D.~Mermin
\item {\em Principles of the Theory of Solids},  J.M.~Ziman
\item {\em The Oxford Solid State Basics}, S.H.~Simon
\end{itemsquish}

\subsection*{Course Objectives: }

\begin{itemsquish}
\item Understand the basic properties of crystalline solids and other condensed-matter systems
\item Learn methods for describing the electronic and lattice structure of solids
\item Understand the theory behind experimental probes of solid-state systems
\end{itemsquish}


\subsection*{Lecture organization and topics: }

\noindent Course material will be presented through lectures on the blackboard, and occasionally with slides. Notes will be posted on the class website as well as further readings. Topics will include: 

\begin{center}
\begin{tabular}{ll}
  topic   & G \& P Ch.\\
\hline
  Crystal lattices   & II \\
  Semiclassical treatment of electrons   & III  \\
  Band theory of crystals & V \\
  Electronic structure properties and methods    & II,VI  \\
  Electronic excitations (excitons, plasmons) & VII \\
  Dielectric screening & VII  \\
  Electron-lattice coupling, phonons   & VIII, IX   \\
 X-ray, neutron scattering   & X   \\
 Optical and transport properties   & XI-XIV   \\
  Magnetic properties   & XV-XVII  \\
  Superconductivity  & XVIII   \\
\hline
\end{tabular}
\end{center}
 

\subsection*{Homework:}

\noindent Homework will be assigned every 1-2 weeks via the course webpage. Assignments will include conceptual, analytical, and \emph{computational} exercises (only very light coding will be required via, e.g., mathematica, matlab, python, or another language of your choice). Assignments will be submitted via Blackboard. \textbf{\textit{Homework will be worth 50 \% of the final grade.}}


\subsection*{Exams:}

\noindent There will be one midterm exam (date TBD) and a final exam (during the assigned final exam period). The final will be cumulative. \textbf{\textit{The midterm will be worth 20 \% of the final grade, and the final exam will be worth 30 \% of the final grade.}}



\subsection*{Americans with Disabilities Act: }

\noindent If you have a physical, psychological, medical or learning
disability that may impact your course work, please contact Disability
Support Services, ECC (Educational Communications Center) Building,
Room 128, (631) 632-6748. They will determine with you what
accommodations, if any, are necessary and appropriate. All information
and documentation is confidential.

\subsection*{Academic Integrity: }

\noindent Each student must pursue his or her academic goals honestly
and be personally accountable for all submitted work. Representing
another person's work as your own is always wrong. Faculty are
required to report any suspected instances of academic dishonesty to
the Academic Judiciary. Faculty in the Health Sciences Center (School
of Health Technology \& Management, Nursing, Social Welfare, Dental
Medicine) and School of Medicine are required to follow their
school-specific procedures. For more comprehensive information on
academic integrity, including categories of academic dishonesty,
please refer to the academic judiciary website at
\url{http://www.stonybrook.edu/commcms/academic_integrity/}

\subsection*{Critical Incident Management: }

\noindent Stony Brook University expects students to respect the
rights, privileges, and property of other people. Faculty are required
to report to the Office of Judicial Affairs any disruptive behavior
that interrupts their ability to teach, compromises the safety of the
learning environment, or inhibits students' ability to learn.  Faculty
in the HSC Schools and the School of Medicine are required to follow
their school-specific procedures.


\subsection*{Electronic Communication: }

\noindent Email to your University email account is an important way
of communicating with you for this course.  For most students the
email address is `{\tt firstname.lastname@stonybrook.edu}'.
%, and the account can be accessed here.
{\em It is your responsibility to read your email received at this
  account.}  For instructions about how to verify your University
email address see this: \\[0.25em]
\url{http://it.stonybrook.edu/help/kb/checking-or-changing-your-mail-forwarding-address-in-the-epo}
\\[0.25em]
%
You can set up email forwarding using instructions here: \\[0.25em]
\url{http://it.stonybrook.edu/help/kb/setting-up-mail-forwarding-in-google-mail}
\\[0.25em]
%
If you choose to forward your University email to another account, we
are not responsible for any undeliverable messages.

\subsection*{Religious Observances: }

\noindent See the policy statement regarding religious holidays at
\\[0.25em] {
  \url{http://www.stonybrook.edu/registrar/forms/RelHolPol\%20081612\%20cr.pdf}}
\\[0.25em]
%
Students are expected to notify the course professors by email of
their intention to take time out for religious observance.  This
should be done as soon as possible but definitely before the end of
the `add/drop' period.  At that time they can discuss with the
instructor(s) how they will be able to make up the work covered.

\end{document}
